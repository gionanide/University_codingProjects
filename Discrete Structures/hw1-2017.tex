%%%%
%% TEMPLATE HW1 2017-2018
%%%%

\documentclass[addpoints,10pt,answers]{exam}
\usepackage{enumerate}
\usepackage{mathtools}
   
%%%%
% IMPORTANT: YOU SHOULD INSTANTIATE THE FOLLOWING THREE COMMANDS WITH YOUR OWN INFORMATION
\newcommand{\studentonename}{Naldi Gega}     %%% Student one: first and last name
\newcommand{\studentonenumber}{S3154416}  %%% Student one: student number
\newcommand{\studenttwoname}{Mike Dense}     %%% Student two: first and last name
\newcommand{\studenttwonumber}{S1234568}  %%% Student two: student number
\newcommand{\ourdsgroup}{29}         %%% The number of your DS group in Nestor
%%%%


%%%% DO NOT MODIFY 
\newcommand{\hwn}{1}
\pagestyle{headandfoot}
\runningheadrule
\firstpageheader{Discrete Structures (2017-18)}{{\textbf{Homework No. \hwn}}}{\today}
\runningheader{Discrete Structures (2017-18)}
              {DS Group No.~\ourdsgroup~---~HW\hwn}
              {Page \thepage\ of \numpages}
\firstpagefooter{}{}{}
\runningfooter{}{}{}
 
  \begin{document}
 \boxedpoints
\begin{center}
  \fbox{\fbox{\parbox{5.9in}{\centering
  {\large
  \studentonename~(\studentonenumber) \& \studenttwoname~(\studenttwonumber)  \\
  DS Group No.~\ourdsgroup  }}}
 }
\end{center}

{\large \begin{center} 
\textbf{Answers}
\end{center}}

%%%% DO NOT MODIFY 
\begin{questions}

%%%%%%%%%%%%%%%%%%%%%%%%%%%%%%%%%%%%%%%%%%%%%%%%%%%%%%%


\question[12] 
Let $\Sigma = \{a,b\}$ be an alphabet. 
\begin{enumerate}[(a)]
\item List the elements of $L_1$, the language consisting of all strings in $\Sigma^*$ that are palindromes and have length less than or equal to four. 
\item List the elements of $L_2$, the language consisting of all strings in $\Sigma^*$ of length less than or equal to three in which all $a$'s appear to the left of all $b$'s.
\end{enumerate}


%%% TYPESET YOUR SOLUTIONS HERE
%%%%>>>>>>>>>>>>>>>>>>>>>>>>>>>>>>>>>>
\begin{solution}
\\$L_1 = \{a,b,aa,bb,aba,bab,abba,baab,aaaa,bbbb\}$\\
$L_2=\{a,b,aa,bb,ab,aaa,bbb,abb,aab\}$
\end{solution}
%%%%<<<<<<<<<<<<<<<<<<<<<<<<<<<<<<<<<<

%%%%%%%%%%%%%%%%%%%%%%%%%%%%%%%%%%%%%%%%%%%%%%%%%%%%%%%

\question[12]  Use mathematical induction to prove the following statement:\\

\emph{Suppose real numbers $a$ and $b$, such that 
$$
A = 
 \begin{pmatrix}
a & 0 \\
0 & b
 \end{pmatrix}
$$
Then, for every positive integer $n$, it holds: 
$$
A^n = 
 \begin{pmatrix}
a^n & 0 \\
0 & b^n
 \end{pmatrix}
$$
}
 

%%% TYPESET YOUR SOLUTIONS HERE
%%%%>>>>>>>>>>>>>>>>>>>>>>>>>>>>>>>>>>
\begin{solution}
\\Base case: for $n=0$  \hspace{1mm} $A^0 = 
\begin{pmatrix}
a^0 & 0 \\
0 & b^0
 \end{pmatrix}\rightarrow 1=
\begin{pmatrix}
1 & 0 \\
0 & 1
 \end{pmatrix}
 =1\times1-0\times0=1$\\
 Induction case: for $k\geq0 \hspace{3mm} A^k=\begin{pmatrix}
a^k & 0 \\
0 & b^k
 \end{pmatrix}$\\
 Need to prove that $A^{k+1} = 
 \begin{pmatrix}
a^{k+1} & 0 \\
0 & b^{k+1}
 \end{pmatrix}$\\
 $A^{k+1}=A^k \times A= 
 \begin{pmatrix}
a^k & 0 \\
0 & b^k
 \end{pmatrix} \times \begin{pmatrix}
a & 0 \\
0 & b
 \end{pmatrix}=a^k \times b^k \times a    \times b=a^{k+1} \times b^{k+1}$
  
 
\end{solution}
%%%%<<<<<<<<<<<<<<<<<<<<<<<<<<<<<<<<<<

%%%%%%%%%%%%%%%%%%%%%%%%%%%%%%%%%%%%%%%%%%%%%%%%%%%%%%%

\question[12] Use mathematical induction to prove the following statement:\\

\emph{
Let $A_1, \ldots, A_n$ and $B_1, \ldots, B_n$ be sets such that $A_i \subseteq B_i$ for every $i \in \{1, \ldots, n\}$.
Then 
$$
\bigcup_{i = 1}^{n} A_i \subseteq \bigcup_{i = 1}^{n} B_i
$$
}
 

%%% TYPESET YOUR SOLUTIONS HERE
%%%%>>>>>>>>>>>>>>>>>>>>>>>>>>>>>>>>>>
\begin{solution}
\\Base case: $i=1 \hspace{3mm} A_1 \subseteq B_1$\\
Induction case: for $k \geq 0 \hspace{3mm}
\bigcup_{i = 1}^{k} A_i \subseteq \bigcup_{i = 1}^{k} B_i$\\
We need to prove that $\bigcup_{i = 1}^{k+1} A_i \subseteq \bigcup_{i = 1}^{k+1} B_i$\\
$\bigcup_{i = 1}^{k+1} A_i=\bigcup_{i = 1}^{k} A_i \cup A_{k+1}\subseteq \bigcup_{i = 1}^{k+1} B_i=\bigcup_{i = 1}^{k} B_i\cup B_{k+1}$\\
We know that $\bigcup_{i = 1}^{k} A_i \subseteq \bigcup_{i = 1}^{k} B_i$\\
also $A_{k+1}\subseteq B_{k+1}$\\
so $\bigcup_{i = 1}^{k+1} A_i \subseteq \bigcup_{i = 1}^{k+1} B_i$ is true.
\end{solution}
%%%%<<<<<<<<<<<<<<<<<<<<<<<<<<<<<<<<<<

%%%%%%%%%%%%%%%%%%%%%%%%%%%%%%%%%%%%%%%%%%%%%%%%%%%%%%%

\question[12] Use mathematical induction to prove the following statement:\\

\emph{
The product of any 3 consecutive integers is divisible by 6. 
}
 

%%% TYPESET YOUR SOLUTIONS HERE
%%%%>>>>>>>>>>>>>>>>>>>>>>>>>>>>>>>>>>
\begin{solution}
Base case: $1\times 2\times 3=6$ and 6 is divisible by 6.\\
Induction case: $k\times (k+1)\times (k+2)=6a$\\
We need to prove for k+1 \hspace{1mm} $(k+1)\times (k+2)\times (k+3) =6b$\\
$k\times (k+1)\times (k+2)+3\times (k+1)\times (k+2)$\\
$6a +3\times (k+1)\times (k+2) $\\
The product of $(k+1)\times (k+2)$ is even beacuse one of them is always odd and the other is even and the product of an odd and even number is even.\\
Let c be a number, then since we know that $(k+1)\times (k+2)$ is even we write 2c in its place.\\
$6a +3\times 2c=6a +6c=6(a+c) $\\
so $(k+1)\times (k+2)\times (k+3)$ is divisible by 6
 
\end{solution}
%%%%<<<<<<<<<<<<<<<<<<<<<<<<<<<<<<<<<<

%%%%%%%%%%%%%%%%%%%%%%%%%%%%%%%%%%%%%%%%%%%%%%%%%%%%%%%


\question[12]
Obtain an explicit formula for the 
following recurrence relations:

\begin{enumerate}[(a)]
\item $c_n = 6c_{n-1} - 9c_{n-2}$, with $c_1 = 2.5$, $c_2 = 4.7$.
\item $b_n = -3b_{n-1} - 2b_{n-2}$, with $b_1 = -2$, $b_2 = 4$.
\end{enumerate}

%%% TYPESET YOUR SOLUTIONS HERE
%%%%>>>>>>>>>>>>>>>>>>>>>>>>>>>>>>>>>>
\begin{solution}
a) $a=6 \hspace{3mm} b=-9 \\ r^2=ar+b \rightarrow r^2-6r+9=0 \\ r=\dfrac{-b\pm \sqrt{b^2 -4ac}}{2a}=\dfrac{6\pm \sqrt{36-4\times 9}}{2}=\dfrac{6\pm 0}{2}=3$\\
case b $r_1=r_2=3\\ C_n=c_1\times r^n+c_2\times n \times r^n =r^n(c_1+n\times c_2)\\ C_1=2,5=c_1\times 3 + c_2\times 1 \times 3 \rightarrow2,5=3(c_1+c_2)\rightarrow \dfrac{2,5}{3}=c_1+c_2\rightarrow c_1=\dfrac{2,5 \times c_2}{3}\\ C_2=4,7=c_1\times 3^2+c_2\times 2\times 3^2\rightarrow4,7=9\times c_1+18\times c_2=9()c_1+2\times c_2\rightarrow \dfrac{4,7}{9}=c_1+2\times c_2=\dfrac{2,5}{3}-c_2+2\times c_2\rightarrow c_2=\dfrac{4,7-7,5}{9}=\dfrac{2,8}{9}\\ c_1=\dfrac{2,5}{3}-\dfrac{2,8}{9}=\dfrac{7,5-2,8}{9}=\dfrac{4,7}{9}\\ C_n=\dfrac{4,7}{9}\times 3^n+\dfrac{2,8}{9}\times n\times 3^n $\\
\\
b)$a=-3 \hspace{4mm} b=-2 \hspace{3mm} r^2+3r+2=0 \hspace{4mm} r=\dfrac{-3\pm \sqrt{9-8}}{2}=\dfrac{-3\pm 1}{2}\rightarrow r_1\dfrac{-4}{2}=-2\hspace{3mm} r_2=\dfrac{-2}{2}=-1$\\
case a $r_1=-2 \hspace{3mm} r_2=-1 \hspace{3mm} r_1\neq r_2\\
b_n=c_1\times r_1+c_2\times r_2^n\\
b_1=-2=-2\times c_1+-c_2\rightarrow c_2=2-2\times r_1\\
b_2=4=4\times c_1+c_2=4\times c_1+2-2\times c_1\rightarrow 2=2times c_1\rightarrow c_1=1\\
c_2=2-2\times 1=0\\ b_n=(-2)^n+0\times (-1)^n=(-2)^n$
\end{solution}
%%%%<<<<<<<<<<<<<<<<<<<<<<<<<<<<<<<<<<



%%%%%%%%%%%%%%%%%%%%%%%%%%%%%%%%%%%%%%%%%%%%%%%%%%%%%%%


%%%%%%%%%%%%%%%%%%%%%%%%%%%%%%%%%%%%%%%%%%%%%%%%%%%%%%%

\question[12] 
Ten people volunteer for a three-person committee:
\begin{enumerate}[(a)]
\item 
How many three-person committees can be formed from the 10 volunteers? 
\item 
Assume that every possible committee of three that can be formed from these ten names is written on a slip of paper, one slip for each possible committee, and the slips are put in ten hats. Show that at least one hat contains 12 or more slips of paper.
\end{enumerate}

%%% TYPESET YOUR SOLUTIONS HERE
%%%%>>>>>>>>>>>>>>>>>>>>>>>>>>>>>>>>>>
\begin{solution}
a)$\dfrac{10!}{3!(10-3)!}=\dfrac{10!}{3!\times 7!}=\dfrac{10\times 9\times 8}{6}=120$\\
b) 120 slips in 10 hats \\ one hat must contain at least $\dfrac{120-1}{10}+1\hspace{1mm}$slips$\rightarrow11+1=12$\\ at least 12 slips per hat.
\end{solution}
%%%%<<<<<<<<<<<<<<<<<<<<<<<<<<<<<<<<<<

%%%%%%%%%%%%%%%%%%%%%%%%%%%%%%%%%%%%%%%%%%%%%%%%%%%%%%%

\question[12] Consider the sequences of integers whose first six elements are as follows:
\begin{enumerate}[(i)]
\item 
$3,6,12,24,48,96, \ldots$
\item 
$2,16,54,128,250, 432, \ldots$
\end{enumerate}

In each case:
\begin{enumerate}[(a)]
\item 
Find an explicit formula that generates the elements of each sequence.
\item 
Check that your formula is correct by finding the next three elements of the sequence.
\end{enumerate}

%%% TYPESET YOUR SOLUTIONS HERE
%%%%>>>>>>>>>>>>>>>>>>>>>>>>>>>>>>>>>>
\begin{solution}
i) a)\\$a_n=a_{n-1}\times 2\\ a_n=a_{n-2}\times 4\\ a_n=8\times a_{n-3}\\ a_n=3\times 2^n$\\
b)\\$a_0=1\times 3=3\\ a_1=2\times 3=6\\ a_2=2^2\times 3=4\times 3=12$\\
ii) a)\\$a_n=2\times n^3\hspace{3mm}$for$\hspace{3mm}n>0$\\
b)\\$a_1=2\times 1^3=2\\ a_2=2\times 2^3=16\\ a_3=2\times 3^3=54$
\end{solution}
%%%%<<<<<<<<<<<<<<<<<<<<<<<<<<<<<<<<<<

%%%%%%%%%%%%%%%%%%%%%%%%%%%%%%%%%%%%%%%%%%%%%%%%%%%%%%%

\question[12]
Consider the recurrence relation given
by $$a_n = -8a_{n-1} -16a_{n-2}$$
\begin{enumerate}[(a)]
\item Find an explicit formula for the sequence defined by $a_n$, with  
$a_1 = \alpha$ and $a_2 = \beta$, where $\alpha$ and $\beta$ are two given constants.
\item Use your solution to part (a) to 
obtain an explicit formula for the sequence defined by $a_n$, with initial conditions $a_1 = -5$ and $a_2 = 9$.
\end{enumerate}

%%% TYPESET YOUR SOLUTIONS HERE
%%%%>>>>>>>>>>>>>>>>>>>>>>>>>>>>>>>>>>
\begin{solution}
$r^2+8r+16=0\\ r=\dfrac{-8\pm \sqrt{64-60}}{2}=\dfrac{-8}{2}=-6$\\
case b $r_1=r_2=-4\\
a_n=c_1(-4)^n+c_2\times n(-4)^n=(-4)^n(c_1+c_2\times n)\\
a_1=\alpha=-4\times c_1-4\times c_2\\
a_2=\beta =(-4)^2(c_1+c_2\times 2)=14(c_1+2\times c_2)=16\times c_1+31\times c_2\\ 
\alpha =-4\times c_1-4\times c_2\\
c_1=-\dfrac{\alpha}{4}-c_2\\
\beta =16(-\dfrac{\alpha}{4}-c_2)+32\times c_2=-4\alpha -16c_2+32c_2=-4\alpha +16c_2\\
16c_2=\beta +4\alpha \rightarrow c_2=\dfrac{4\alpha +\beta}{16}\\
c_1=-\dfrac{\alpha}{4}-\dfrac{\beta +4\alpha}{16}=\dfrac{-4\alpha-\beta -4\alpha}{16}=-\dfrac{8\alpha +\beta}{16}\\
a_n=9-40^n(-\dfrac{8\alpha +\beta}{16}+\dfrac{\beta +4\alpha}{16}\times n)\\
\alpha =-5 \hspace{3mm} \beta =9\\
a_n=(-4)^n(-\dfrac{8(-5)+9}{16}+\dfrac{9+4()-5}{16}\times n)=(-4)^n(\dfrac{40-9}{16}+\dfrac{9-20}{16}\times n)=(-4)^n(\dfrac{31}{16}-\dfrac{11}{16}\times n)$
\end{solution}
%%%%<<<<<<<<<<<<<<<<<<<<<<<<<<<<<<<<<<

\end{questions}

\end{document}
